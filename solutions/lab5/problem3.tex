\section*{Problem 3}
An equation of plane can be given on a \emph{Point-normal form (PNF)}. To fulfill this form
we need a normal of the plane, $\hat{n}$ and a point, $B$ that lies on the plane.
Then we can find any point $R$ on the plane by solving the equation
\begin{equation}
    \left( R - B \right) \cdot \hat{n} = 0
    \label{eq:pnf}
\end{equation}
where the vector $R-B$ must be orthogonal with the normal $\hat{n}$ using the orthogonal property of the
dot product.\\

There are many ways to give a formula for a line (or a ray). One is the \emph{Parametric form}.
Given one point $p$ on the line (or an initial point of a ray) and the directional vector $\vec{c}$ of the line.
We can find any point $p'$ on the line by scaling the vector $\vec{c}$ with some parameter $t$ and translate
the point $p$ with that vector.
\begin{equation}
    p' = p + t\vec{c}
    \label{eq:parametric}
\end{equation}

In the following problem we wish to find where a given line hits a plane, that is the point of hit.
We will call that point $p_{hit}$. We are also interested in the time of hit, that we call $t_{hit}$.

With our hypothetical line (that is hitting the plane) and from (\ref{eq:parametric}) we know that
\begin{equation}
    p_{hit} = A + ct_{hit}
    \label{ref:phit}
\end{equation}
and from (\ref{eq:pnf}) we know that
\begin{equation}
    \left( p_{hit} - B \right) \cdot \hat{n} = 0
    \label{ref:step2}
\end{equation}
where the point $p_{hit}$ lies on the plane (where the line hits the plane) and the vector $\left( p_{hit}-R \right)$ is therefore orthogonal to the normal of the plane.

By combining (\ref{ref:phit}) and (\ref{ref:step2}) we get
\begin{equation}
    \left( A+ct_{hit} -B \right) \cdot \hat{n} = 0
    \label{eq:step3}
\end{equation}
We can now rewrite (\ref{eq:step3}) as follows (the aim is to isolate $t_{hit}$):
\begin{eqnarray*}
    ((A-B) + ct_{hit}) \cdot \hat{n} &=& 0\\
    (A-B) \cdot \hat{n} + ct_{hit} \cdot \hat{n} &=& 0\\
    ct_{hit} \cdot \hat{n} &=& (B-A) \cdot \hat{n}\\
    t_{hit}(c \cdot \hat{n}) &=& (B-A) \cdot{n}\\
    t_{hit} &=& \frac{(B-A) \cdot \hat{n}}{c \cdot \hat{n}} 
\end{eqnarray*}
From this we can fill in (\ref{ref:phit}):
\begin{equation}
    p_{hit} = A +c\left(  \frac{(B-A) \cdot \hat{n}}{c \cdot \hat{n}} \right) 
\end{equation}
and we can use this formula to find where our line hits a plane.

\subsection*{Solving the problem using $t_{hit}$ and $p_{hit}$}
We are given three points on a plane:
\begin{equation*}
    p_1 = (3,2,4), \quad p_2 = (4,3,0), \quad p_3 = (5,2,3)
\end{equation*}
We then have a ray, starting at point $A=(-1,-2,3)$ which has the direction $\vec{c}=(1,5,3)$. Let us find
$p_{hit}$ and $t_{hit}$ where the line hits the plane.

First we need the normal vector of the plane
\begin{eqnarray*}
    \hat{n} &=& (p_2 - p_1) \times (p_3 - p_1)\\
    \hat{n} &=& (1,1,-4) \times (2,0,-1)\\
    \hat{n} &=& (-1,-7,-2)
\end{eqnarray*}
We can now use the $t_{hit}$ formula the we derived above. We select $B=p_1$.
\begin{eqnarray*}
    t_{hit} &=& \frac{(B-A) \cdot \hat{n}}{c \cdot \hat{n}}\\
    t_{hit} &=& \frac{ (4,4,1)\cdot(-1,-7,-2) }{(1,5,3)\cdot(-1,-7,-2)}\\
    t_{hit} &=& \frac{34}{42}
\end{eqnarray*}
We now know $t_{hit}$ and we can find the point where the ray hits the plane
with $p_{hit}$.
\begin{eqnarray*}
    p_{hit} &=& A+ct_{hit}\\
    p_{hit} &=& (-1, -2,3)+\left( \frac{34}{42}(1,5,3) \right)\\
    p_{hit} &=& \left( -1+\frac{34}{42}, -2\frac{170}{42}, \frac{102}{42} \right)
\end{eqnarray*}