\documentclass[11pt,a4paper]{article}
\usepackage[utf8]{inputenc}
\usepackage[T1]{fontenc}
\usepackage[icelandic]{babel}
\usepackage{amsmath, amsthm, amssymb, amsfonts}  %Whoo Math.
\usepackage{geometry}
\usepackage{graphicx}
\usepackage{listingsutf8}
\usepackage[usenames,dvipsnames,svgnames,table]{xcolor}
\usepackage{tcolorbox}
\usepackage{fancyhdr}
\usepackage{hyperref}
\tcbuselibrary{listings}

%listings settings
\lstset{
    numberstyle=\small\ttfamily,
    numbersep=4pt,
    numbers=left,
    basicstyle=\small\ttfamily,,
    showstringspaces=false,
    escapebegin=\begin{text},
    escapeend=\end{text},   
    language=java,
    captionpos=b,
frame=single
}

\begin{document}
%----------------------------------------------------------
%title
\title{\textbf{Lab assignment 2}}
\author{hlysig}
\date{}
\maketitle
\section{Introduction}
This document contains the second lab class assignment for the course
T-511-TGRA (Computer graphics). The source code for this document is hosted on
Github (hlysig/tgra-2013). Students are encouraged to submit pull request for
errors and improvements.

\section{Coordinate systems}
In the first lab class we discussed briefly how we defined the coordinate
system for our applications. We placed similar code as in listing
(\ref{setting_up_coordinates}) into our display, reshape or init function.
This code should look similar to the one you write in Java LibGDX, C/C++ or any
other programming language using OpenGL.

\begin{lstlisting}[caption=Setting up the coordinat system from first lab class, label=setting_up_coordinates]
// Load the Project matrix. Next commands will be 
// applied on that matrix.
glMatrixMode(GL_PROJECTION);

// Fill the projection matrix with the identity matrix
glLoadIdentity();

// Set up a two-dimensional orthographic viewing region.
gluOrtho2D(0, width, 0, height);

// Set up affine transformation of x and y from world coordinates 
// to window coordinates
glViewport(0, 0, width, height);

// Set the Modelview matrix back.
glMatrixMode(GL_MODELVIEW_MATRIX); 
\end{lstlisting}

Though we haven't discussed what these functions do (glMatrixMode and
glViewport), we stated that they would set up a coordinate system and a
viewport mapping to the window coordinates. That is, the coordinate system for
our application will be equal to the window that we are using.

For some applications this is good but in some cases you might want to control how
the coordinate system is represented and furthermore you might want to control
how this coordinate system is mapped, or projected on the screen. This is most often
referred to as \emph{Window to viewport transformation} (W2V) and you will hear a great deal
about that term in this course.

Without going heavily into the math of W2V (there will be a lecture on that
topic later in the course), let us identify a simple problem where the need
rises to use a different coordinate system and solve it with the tools that we
have.

\section{Math plotter}
Let us write a simple OpenGL application that allows us to visualize a given
mathematical function on a given interval. Let us plot the function
(\ref{ex:sinx}) on the interval $[0, 10]$.

\begin{equation}\label{ex:sinx}
f(x) = sin(x)
\end{equation}

We can reuse our application from the first lab class (The box world) to plot
this function. We begin by writing a function that returns a buffer of
$\mathbb{R}^2$ points where each point is on the line generated from
(\ref{ex:sinx}) and the distance between the points is some small number $\phi$
that we select. This function can be implemented similar to listing
(\ref{line_points}).

\begin{lstlisting}[texcl, caption=Function for calculating porints on line on some interval for sin(x), label=line_points]
private float[] createPlotPoints(float lower, 
                                float higher, float step) {		
  int buffer_size = (int)  (Math.ceil((higher-lower)/step)*2) + 2;
  float[] buffer = new float[buffer_size];
  int curr_index = 0;
  for(float x = lower; x<higher; x+=step) {
    buffer[curr_index] = x;
	buffer[curr_index+1] = (float)Math.sin(x);
	curr_index +=2;
  }
  return buffer;
}
\end{lstlisting}


The first parameter, lower, is the left side fringe point of the curve and the
higher parameter is the right side fringe point. The last parameter is our
$\phi$ value, that is, the distance between points on the line that we are
plotting. Larger value for $\phi$ gives a more solid line. 

Now we can use this function to create vertex buffer for the sine curve
that we can then plot. Listing (\ref{code:sinebuffer}) is an example of how
we might do this in code.


\begin{lstlisting}[texcl,label=code:sinebuffer, caption=Create vertex buffer for our sine function. ]
float[] f = this.create_points(0f, 10f, 0.005f);
this.number_of_points = f.length / 2;
this.vertexBuffer = BufferUtils.newFloatBuffer(f.length);		
this.vertexBuffer.put(f);
this.vertexBuffer.rewind();
Gdx.gl11.glVertexPointer(2, GL11.GL_FLOAT, 0, this.vertexBuffer);
\end{lstlisting}


Note that we have a member variable for the number of points in the buffer. We
use this variable to know how many points we have to draw.

Now we are ready to draw our the sine curve. We can write a code similar to the
one in listing (\ref{code:drawcine}) into our display or render function.


\begin{lstlisting}[texcl, label=code:drawcine, caption=Drawing the sine curve]
private void display() {
    Gdx.gl11.glClear(GL11.GL_COLOR_BUFFER_BIT);
    Gdx.gl11.glMatrixMode(GL11.GL_MODELVIEW);
    Gdx.gl11.glLoadIdentity();
    Gdx.gl11.glColor4f(1f, 1f, 1f, 1f);
    
    Gdx.gl11.glPushMatrix();
    Gdx.gl11.glDrawArrays(GL11.GL_POINTS, 0, this.number_of_points);
    Gdx.gl11.glPopMatrix();	
}
\end{lstlisting}

Now if you have written this application, or know the nature of the sine
function, then you might have found out that there are two major problems with
our solution. First, we are plotting on the interval $[0, 10]$ using screen
coordinates. This means that we are drawing the sine function on the first 10
pixels on the screen. The plot will be rather small - tiny would be a better
word to describe it. Second, when $x$ is close to $3$ and $9$ the $y$ value are
negative. That is, part of the curve will not be visible with current setup
where part of it is outside of our window.

This is no good and we need to fix this. To begin with we brute force this
problem by applying a form of a affine transformation where we scale and
translate the pixels that we intend to draw so we can see our sine curve on the
screen.

When we have managed to do that we alter our code to use the internal functions
from OpenGL that solves this for us. Why go through all this hassle and why not
just introduce the internal OpenGL functionality that solves this for us? The
brute force method that we introduce here is similar to the one used by OpenGL.
It is good to know how OpenGL solves things to be able to apply them
effectively and efficiently in our applications.

Now let us translate the $x$-values of our drawing to use the entire screen
(not only the first 10 pixels). We have the range from 0 to 10. We can do it by
scaling the x value with respect to the interval and the width of the screen,
as we do in (\ref{eq:scaled_x}).

\begin{equation}\label{eq:scaled_x}
x_s = \frac{x \times \textnormal{screen width}}{10.0} 
\end{equation}

Let us convince ourself that this is what we need. When $x=0$ then
(\ref{eq:scaled_x}) becomes 0 and when $x=10$ it becomes equal to the screen
width. That is the interval $[0,10]$ will be scaled to $[0, \textnormal{screen width}]$.
Awesome!

Now let us scale and transform the $y$ values. Somewhere in our past we
learned that the $y$ values for the sine function are always between $-1$ and
$1$. Thus for the $y$ value we scale it to the center of the screen and
translate it by $1$. This can be preformed as in (\ref{eq:scaled_y}).

\begin{equation}\label{eq:scaled_y}
y_s = \frac{(y +1.0) \times \textnormal{screen height}}
           {\textnormal{2.0}}
\end{equation}

The formulae \ref{eq:scaled_x} and \ref{eq:scaled_y} can be written formally as follows.
\begin{eqnarray}
s_x &=& Ax + B\\
s_y &=& Cy + D
\end{eqnarray}

No worries. You will hear more from these formulae later in the course. Let us
just trust that they are right and correct. We are in no math mood today.

We can then change our function from listing (\ref{line_points}) to favor our
new discovery for scaling and translating the $x$ and $y$ values for each point
for the sine function to fit on the screen as listing (\ref{refining-points-scale}).

\begin{lstlisting}[caption=Creating vertex buffer for the sine function using scaling and transformation, label=refining-points-scale]
private float[] create_points(float lower, float higher, float step){
  float A,B,C,D;

  A = this.screen_x / (higher-lower);
  B = 0;
  C = this.screen_y / 2.0f;
  D = C;

  int buffer_size = (int)  (Math.ceil((higher-lower)/step)*2) + 2;

  float[] buffer = new float[buffer_size];
  int curr_index = 0;
  for(float x = lower; x<higher; x+=step){
	buffer[curr_index] = A*x + B;
	buffer[curr_index+1] = C * ((float) Math.sin(x)) + D;
	curr_index +=2;
  }
  return buffer;
}
\end{lstlisting}

After these changes you should see your plot on the screen. Amazed? I hope so.
Next we look at the functionality that OpenGL provides for us so we don't have
to deal with this awful scaling and translation ourself!

\subsection{The World Window}
When drawing an OpenGL scene we can enter vertices with any coordinates, in a
continuous space of unlimited size. Our screen, however is of a fixed size and
we need to control which part of our endless virtual world we project onto it.
In 2D we define the World Window. The World Window simply has values for it's
left, right, bottom and top edges and anything within those boundaries will be
projected onto the screen. Anything outside will be cut away. To set up a World
Window we use the function:
\begin{lstlisting}
gluOrtho2D(left, right, bottom, top)
// For libGDX:
Gdx.glu.gluOrtho2D(Gdx.gl10, left, right, bottom, top);
\end{lstlisting}
The values for the edges can be any floating point numbers and they need have
no connection at all with the pixels or even the width/height ratio of the
screen we‘ll be projecting it on. If we imagine our $1.0f$ unit represents a
meter and we‘ll be drawing microscopic objects then we can put in numbers like
$0.0001$ and if we are drawing solar systems we can use $1000000.0f$ and call each
unit $1000$ kilometers. This is our world and we can set it up any way we want.
For the case of the sine function in our ``sine world'' the world window might
be from $0$ to $10$ and $-1$ to $1$.

\section{The Viewport}
The viewport is the area on the screen that we will be projecting our scene
onto. The default Viewport is simply the whole drawable canvas of our program
window but we can control this as well. The OpenGL function we use is:
\begin{lstlisting}
glViewport(x, y, width, height)
//libGDX:
Gdx.gl11.glViewport(x, y, width, height);
\end{lstlisting}
$(x,y)$ refers to the lower left corner of the area we want to project onto and
width and height are simply the width and hight (in pixels) of that area. These
variables are integer numbers and must describe a viewport that lies within the
drawable canvas of our program window.When our scene is projected onto the
screen the lower left corner of the World Window is fitted to the lower left of
the Viewport and the upper right corners aligned as well.

Within the same frame we can call both the glViewport and the Ortho2D functions
as often as we wish and are able to work with different viewports and world
windows any way we like. Common applications are split screen multiplayer games
and little map views in one corner of a main game screen.

\section{Access to the program window}
LibGDX gives us a way to retrieve the current size of the drawable area of our
program window through the functions:
\begin{lstlisting}
Gdx.graphics.getWidth();
Gdx.graphics.getHeight();
\end{lstlisting}
This allows us to taylor our World Window and Viewport to that area and helps
us properly work with different full screen resolutions, i.e. on different
types of Android phones.

\section{Last refinement of our sine program}
Before we leave our sine application for ever, let us make one last improvement
using the above mentioned Window and viewport. Revert the function that
we implemented to create the vertex buffer for the sine function to it's
original state (as in listing (\ref{line_points})). Note that we stated that this
implementation had some issues that we discussed earlier. That was not entirerly
correct. To solve that we applied some transformations ourself. Let us use
window to viewport mapping to solve this for us as the following listing.
\begin{lstlisting}
public void resize(int width, int height) {
  Gdx.gl11.glMatrixMode(GL11.GL_PROJECTION);
  Gdx.gl11.glLoadIdentity();
  Gdx.glu.gluOrtho2D(Gdx.gl11, 0, 10, -1, 1);
  Gdx.gl11.glViewport(0, 0, width, height);
  Gdx.gl11.glMatrixMode(GL11.GL_MODELVIEW_MATRIX);    
}
\end{lstlisting}
Note that here we set our world window to $(0,10)$ and $(-1,1)$ than we set the viewport to the
screen size. When you execute the sine application the sine function should be exactly the same -
where OpenGL is applying similar functionality as we implemented beforehand. Awesome!


%% added from kari
\section{Different world windows}
Now let's stop thinking about sine and start doing something more fun. Open up
the project from last week or reuse the box home assignment and let us do some
experiments with windows and viewports. Make a few class variables, just to
keep things clear:
\begin{lstlisting}
float world_left;
float world_right;
float world_bottom;
float world_top;
\end{lstlisting}
Then change the values in your gluOrtho2D call to:
\begin{lstlisting}
Gdx.glu.gluOrtho2D(Gdx.gl10, world_left, world_right, 
               world_bottom, world_top);
\end{lstlisting}
If you are calling these function from reshape then move them to your display
function. This way you won't forget what the values stand for either, as the
parameters in glViewport are a bit different.  Now you can go to your update
function and set values for these new variables. Start with:

\begin{lstlisting}
world_left = 0;
world_right = Gdx.graphics.getWidth();
world_bottom = 0;
world_top = Gdx.graphics.getHeight();
\end{lstlisting}
Make sure your display function is drawing something all over the screen. One
of the first hand-ins should suffice. Then start playing around with the
values. Try displaying only a quarter of the original world. Then try
displaying a portion of that size, but centered in the original world. Try a
larger World Window and try negative values as well. Experiment.  


\section{Set up different Viewports}
Make class variables for the viewport too:
\begin{lstlisting}
float viewport_left;
float viewport_bottom;
float viewport_width;
float viewport_height;
\end{lstlisting}
And somewhere in the display function, close to your Ortho2D method call add:
\begin{lstlisting}
Gdx.gl11.glViewport(viewport_left, viewport_right, 
               viewport_width, viewport_height);
\end{lstlisting}
Also set values for the variables in udpate.
\begin{lstlisting}
viewport_left = 0;
viewport_bottom = 0;
viewport_width = Gdx.graphics.getWidth(); 
viewport_height = Gdx.graphics.getHeight();
\end{lstlisting}
This affects where on the canvas your World Window will be projected. Try to
halve the values for width and height. Then play around with the values for
left and bottom.


\section{Set up multiple Viewports and World Windows}
Now, let's try to display our scene twice, in the same frame but display it
through a different World Window and project it to a different Viewport.  Set
up variables for a second of each (viewport2\_left, world2\_top, etc.).  Nice
initial values for the viewports are:
\begin{lstlisting}
Viewport1_left = 0;
Viewport1_bottom = 0;
Viewport1_width = Gdx.graphics.getWidth() / 2; 
Viewport1_height = Gdx.graphics.getHeight(); 
Viewport2_left = Gdx.graphics.getWidth() / 2; 
Viewport2_bottom = 0;
Viewport2_width = Gdx.graphics.getWidth() / 2; 
Viewport2_height = Gdx.graphics.getHeight(); 
\end{lstlisting}
So we've split the screen in two equal parts, side by side.
Now take your display code (apart from the MatrixMode, Identity, Ortho2D and
Viewport) and move it into a peparate method (maybe private void drawScene() or
something).  In your display function do (in pseudo code here): 
\begin{lstlisting}
Set viewport1
Set world1 
Draw scene 
Set viewport2 
Set world2 
Draw scene 
\end{lstlisting}
Play around with values for the world windows.  Can you set up the left window
so that it's the same size as the viewport so you can draw there pixel for
pixel, then set up a „zoom view“ in the right viewport?  Can you move the right
„zoom view“ around by clicking with the mouse in the left „full view“?
Experiment experiment
experiment!
\end{document}
