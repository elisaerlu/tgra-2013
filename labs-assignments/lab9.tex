\documentclass[12pt]{article}
\usepackage{amsmath}
\title{\textbf{Lab assignment 9}}
\date{}
\author{Hlynur Sigurthorsson\\hlysig@0x.is}
\begin{document}

\maketitle
\thispagestyle{empty}
\section*{Final exam 2012 - Problem 3 - Rasterization}
Describe rasterization in OpenGL. Consider the following questions in your answer
\begin{itemize}
\item What data do you need before the process starts?
\item What does the algorithm calculate?
\item How does the algorithm work?
\item What tests can happen here and where do they fit in?
\end{itemize}


\section*{Final exam 2008 - Problem 6 - Rasterization}
Triangle is sent through the OpenGL and ens up with the screen pixels
\begin{eqnarray*}
p_1 = (5,4), \hspace{0.2cm} p_2 = (3,9),\hspace{0.2cm} p_3 = (14,12)
\end{eqnarray*}
The corners of the triangle has the following color values assigned to them.
\begin{eqnarray*}
c_1 = (0.2,0.2,0.2), \hspace{0.2cm} c_2 = (0.6, 0.6, 0.6),\hspace{0.2cm} c_3 = (0.9, 0.9, 0.9)
\end{eqnarray*}
What will the color value in pixel $(5,6)$ be?


\section*{Final exam 2012 - Problem 4 - a}
A camera is set up to be positioned in $(7,7,7)$ and is looking at point $(2,6,4)$ and has
the up vector $(0,1,0)$.

Find the point of origin and the vectors for the camera coordinate frame.
\end{document}