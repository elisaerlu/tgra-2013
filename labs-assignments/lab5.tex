\documentclass[12pt]{article}
\usepackage{amsmath}
\usepackage{amssymb}
\usepackage{amsthm}
\usepackage[utf8]{inputenc}
\title{\textbf{Lab assignment 5}}
\author{\textbf{T-511-TGRA}}
\date{}

\begin{document}
\maketitle
\section*{Introduction}
This document contains a problem set for Lab class 5 in the course T-511-TGRA.
This document is hosted on Github (https://github.com/hlysig/tgra-2013) for the course. Please be a lad and send us pull requests with fixes if you find any errors in this document.

%%%%%%%%%%%%%%%%%%%%%%%%%%%%%%%%%%%%%%%%%%%%%%%%%%%%%%%%%%%%%%%%%%%%%%%%%
\section*{Problem 1}
\begin{itemize}
    \item[i)] {
    Make 2D transformation matrix $A$ for translating by $(3,5)$. 
     Translate the point $p=(1,2)$ with $A$.
   
    }
    \item[ii)] {
    Make a 2D transformation matrix $B$ for double scaling.
    Transform the point $p=(5,4)$ with $B$.
    }
    \item[iii)] {
        Make a 2D transformation matrix $C$ for $36^\circ$ rotation.

        Transform the point $p=(3,-1)$ with $C$.
           }
    \item[iv)]{
    Make a 2D transformation matrix that applies a combined transformation. The
    combined transformation is as so: translate a coordinate frame first as in
    A, then as in B (relative to the previous coordinate frame) and finally as
    in C (again relative to the previous coordinate frame)
    }
  \end{itemize}
%%%%%%%%%%%%%%%%%%%%%%%%%%%%%%%%%%%%%%%%%%%%%%%%%%%%%%%%%%%%%%%%%%%%%%%%%

%%%%%%%%%%%%%%%%%%%%%%%%%%%%%%%%%%%%%%%%%%%%%%%%%%%%%%%%%%%%%%%%%%%%%%%%%
\section*{Problem 2}
\begin{itemize}
\item[a)] {
How will the \emph{ModelView} matrix be if we execute the following lines.
\begin{enumerate}
        \item \texttt{glLoadIdentity();}
        \item \texttt{gluLookat(5,7,3,4,7,1,0,1,0)}
\end{enumerate}
}

\item[b)]{
        Show the values in the modelview matrix if the following lines of code are executed after the code in a)
        \begin{enumerate}
            \item \texttt{glRotated(30,1,0,0);}
            \item \texttt{glTranslate(0,10,0);}
        \end{enumerate}
}
Please calculate the eye coordinate for the point $p=(1,1,1).$ using the ModelView matrix that you have constructed.
\end{itemize}
%%%%%%%%%%%%%%%%%%%%%%%%%%%%%%%%%%%%%%%%%%%%%%%%%%%%%%%%%%%%%%%%%%%%%%%%%

%%%%%%%%%%%%%%%%%%%%%%%%%%%%%%%%%%%%%%%%%%%%%%%%%%%%%%%%%%%%%%%%%%%%%%%%%
\section*{Problem 3}
We are given three points on a plane:
\begin{equation*}
    p_1 = (3,2,4), \quad p_2 = (4,3,0), \quad p_3 = (5,2,3)
\end{equation*}
We then have a ray, starting at point $A=(-1,-2,3)$ which has the direction $\vec{c}=(1,5,3)$. Let us find
$p_{hit}$ and $t_{hit}$ where the line hits the plane.
%%%%%%%%%%%%%%%%%%%%%%%%%%%%%%%%%%%%%%%%%%%%%%%%%%%%%%%%%%%%%%%%%%%%%%%%%




\end{document}


